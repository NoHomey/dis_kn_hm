\documentclass[12pt]{article}
    
\usepackage[left=2cm,right=2cm,top=1cm,bottom=2cm]{geometry}
\usepackage{amsmath,amsthm}
\usepackage{amssymb}
\usepackage{lipsum}
\usepackage{tikz}
\usepackage[T1,T2A]{fontenc}
\usepackage[utf8]{inputenc}
\usepackage[T1]{fontenc}
\usepackage[bulgarian]{babel}
\usepackage[normalem]{ulem}

\newcommand{\N}{\mathbb{N}}
\newcommand{\R}{\mathbb{R}}

\setlength{\parindent}{0mm}
        
\title{Домашна Работа № 1}
\author{Иво Стратев, Информатика, 2-ри курс, група 3, фак. № 45342}
        
\begin{document}
\maketitle

Нека $F(x) = \frac{2x^2 + 18}{x - 9}$, нека $\lambda \in \R \backslash \{9\}$ и нека е дадена рекуретна редица \\

$\{a_n\}_{n = 1}^\infty$, такава  че: \\

$a_n = \begin{cases}
    \lambda &, \; n = 1\\
    F(a_{n - 1}) &, \; n \geq 2
\end{cases}$ \\\\

Да се изследва редицата $\{a_n\}_{n = 1}^\infty$ за сходимост в зависимост от $\lambda$. \\

Решение: \\

Ако редицата $\{a_n\}_{n = 1}^\infty$ е сходящата, нека $l = \lim_{n \to \infty} \; a_n$. \\

Тогава $\lim_{n \to \infty} \; a_{n + 1} - a_n =  l - l = 0 \implies \\\\
\lim_{n \to \infty} \; a_{n + 1} - a_n = \lim_{n \to \infty} \; F(a_n) - a_n = \\\\
= \lim_{n \to \infty} \; \frac{2a_n^2 + 18}{a_n - 9} - a_n = \lim_{n \to \infty} \; \frac{2a_n^2 + 18}{a_n - 9} - a_n\frac{a_n - 9}{a_n - 9} = \\\\
= \lim_{n \to \infty} \; \frac{2a_n^2 + 18 - a_n^2 + 9a_n}{a_n - 9} = \lim_{n \to \infty} \; \frac{a_n^2 + 9a_n + 18}{a_n - 9} = \frac{l^2 + 9l + 18}{l - 9} = 0 \\\\
\iff l^2 + 9l + 18 = 0 \iff (l + 3)(l + 6) = 0 \iff l = -3 \; \lor \; l = -6$ \\\\

Тогава ако редицата $\{a_n\}_{n = 1}^\infty$ е сходяща нейната граница е или $-6$ или $-3$.

\begin{align*} sign(a_{n + 1} - a_n) = sign\left(\frac{(a_n + 3)(a_n + 6)}{a_n - 9}\right) & ~ & (1) & ~ & \begin{tikzpicture}
    \draw(0,0) -- (4,0);
    \foreach \x in {1,2,3}
        \draw (\x cm,3pt) -- (\x cm,-3pt);
    \draw (1,0) node[below=3pt] {-6};
    \draw (2,0) node[below=3pt] {-3};
    \draw (3,0) node[below=3pt] {9};
    \draw (0.5,0) node[above=3pt] {-};
    \draw (1.5,0) node[above=3pt] {+};
    \draw (2.5,0) node[above=3pt] {-};
    \draw (3.5,0) node[above=3pt] {+};
\end{tikzpicture}
\end{align*} \\\\

$sign(a_{n + 1} - (-6)) = sing(F(a_n) + 6) = sign\left(\frac{2a_n^2 + 18 + 6(a_n - 9)}{a_n - 9}\right) = \\\\\\
= sign\left(\frac{2a_n^2 + 2.9 + 6a_n - 6.9}{a_n - 9}\right) = sign\left(\frac{2a_n^2 + 6a_n - 4.9}{a_n - 9}\right) = sign\left(2\frac{a_n^2 + 3a_n - 18}{a_n - 9}\right) = \\\\
= sign\left(\frac{a_n^2 + 3a_n - 18}{a_n - 9}\right) = sign\left(\frac{(a_n + 6)(a_n - 3)}{a_n - 9}\right) \quad (2) \quad \begin{tikzpicture}
    \draw(0,0) -- (4,0);
    \foreach \x in {1,2,3}
        \draw (\x cm,3pt) -- (\x cm,-3pt);
    \draw (1,0) node[below=3pt] {-6};
    \draw (2,0) node[below=3pt] {3};
    \draw (3,0) node[below=3pt] {9};
    \draw (0.5,0) node[above=3pt] {-};
    \draw (1.5,0) node[above=3pt] {+};
    \draw (2.5,0) node[above=3pt] {-};
    \draw (3.5,0) node[above=3pt] {+};
\end{tikzpicture}$ \\\\\\

$sign(a_{n + 1} - (-3)) = sing(F(a_n) + 3) = sign\left(\frac{2a_n^2 + 18 + 3(a_n - 9)}{a_n - 9}\right) = \\\\
= sign\left(\frac{2a_n^2 + 3a_n + 2.9 - 3.9}{a_n - 9}\right) = sign\left(\frac{2a_n^2 + 3a_n - 9}{a_n - 9}\right)
= sign\left(\frac{(a_n + 3)(a_n - \frac{3}{2})}{a_n - 9}\right) \quad (3) \quad \begin{tikzpicture}
    \draw(0,0) -- (4,0);
    \foreach \x in {1,2,3}
        \draw (\x cm,3pt) -- (\x cm,-3pt);
    \draw (1,0) node[below=3pt] {-3};
    \draw (2,0) node[below=3pt] {$\frac{3}{2}$};
    \draw (3,0) node[below=3pt] {9};
    \draw (0.5,0) node[above=3pt] {-};
    \draw (1.5,0) node[above=3pt] {+};
    \draw (2.5,0) node[above=3pt] {-};
    \draw (3.5,0) node[above=3pt] {+};
\end{tikzpicture}$ \\\\\\

$sign(a_{n + 1} - 9) = sing(F(a_n) - 9) = sign\left(\frac{2a_n^2 + 18 - 9(a_n - 9)}{a_n - 9}\right) = \\\\\\
= sign\left(\frac{2a_n^2 + 2.9 -9a_n + 9.9}{a_n - 9}\right) = sign\left(\frac{2a_n^2 - 9a_n + 11.9}{a_n - 9}\right) = \\\\
= sign(2a_n^2 - 9a_n + 11.9)sign(a_n - 9) = 1.sign(a_n - 9) =  sign(a_n - 9) \quad (4) \quad \begin{tikzpicture}
    \draw(0,0) -- (2,0);
    \draw (1 cm,3pt) -- (1 cm,-3pt);
    \draw (1,0) node[below=3pt] {9};
    \draw (0.5,0) node[above=3pt] {-};
    \draw (1.5,0) node[above=3pt] {+};
\end{tikzpicture}$ \\\\

Нека $\lambda \in (9, \; \infty)$. Тогава от (1) следва, че редицата $\{a_n\}_{n = 1}^\infty$ e монотоно растяща, защото всеки член е строго по-голям от предишния.
От (2) и (3) следва, че $a_2 > -3 > -6$. Тогава $\forall n \in \N \; a_n > -3 > -6$, тоест редицата е неограничена откъдето следва, че $a_k \underset{k \to \infty}{\to} \infty$. \\

Нека $\lambda \in (3, \; 9)$. Тогава от (3) следва, че $a_2 < -3$, а от (2) следва, че $a_2 < -6$. Тогава от (1) следва, че редицата $\{a_n\}_{n = 1}^\infty$ e монотоно намаляваща, защото всеки член е строго по-малък от предишния, заедно с $a_2 < -6 < -3 \implies$ следва,
че редицата $\{a_n\}_{n = 1}^\infty$ e монотоно намаляваща и неограничена, тоест $ a_k \underset{k \to \infty}{\to} -\infty$. \\\\

Нека $\lambda = 3$. Тогава от (2) следва, че $a_2 = -6$ и отново от (2) следва, че $a_3 = -6$. Тогава по индукция, следва че $\forall n \in \N\backslash\{1\} \; a_n = -6$. От където следва, че $a_k \underset{k \to \infty}{\to} -6$. \\\\

Нека $\lambda \in \left(\frac{3}{2}, \; 3\right)$. Тогава от (3) следва, че $a_2 < -3$, а от (2) следва, че $a_2 > -6$. Получаваме $-6 < a_2 < -3$. Тогава чрез индукция по $n$ от (3) и (2) достигаме до неравенството \\

$\forall n \in \N \; -6 < a_n < -3$. От (1) следва, че редицата $\{a_n\}_{n = 1}^\infty$ e монотоно растяща, от което следва, че редицата $\{a_n\}_{n = 1}^\infty$ e монотоно растяща и ограничена отгоре от $-3$ или $a_k \underset{k \to \infty}{\to} -3$. \\\\

Нека $\lambda = \frac{3}{2}$. Тогава от (3) следва, че $a_2 = -3$ и отново от (3) следва, че $a_3 = -3$. Тогава по индукция, следва че $\forall n \in \N\backslash\{1\} \; a_n = -3$. От където следва, че $a_k \underset{k \to \infty}{\to} -3$. \\\\

Нека $\lambda \in \left(-3, \; \frac{3}{2}\right)$. Тогава от (3) следва, че $a_2 > -3$, от (2) следва, че $a_2 > -6$, тоест \\

$a_2 > -3 > -6$, а от (1) следва, че редицата $\{a_n\}_{n = 1}^\infty$ е намаляваща и така по индукция получаваме, че редицата $\{a_n\}_{n = 1}^\infty$ е монотоно намаляваща и ограничена отдолу от $-3$, тогава $a_k \underset{k \to \infty}{\to} -3$. \\\\

Нека $\lambda = -3$. Тогава от (3) следва, че $a_2 = -3$ и отново от (3) следва, че $a_3 = -3$. Тогава по индукция, следва че $\forall n \in \N \; a_n = -3$. От където следва, че $a_k \underset{k \to \infty}{\to} -3$. \\\\

Нека $\lambda \in (-6, \; -3)$. Тогава от (3) следва, че $a_2 < -3$, а от (2) следва, че $a_2 > -6$ тоест $-6 < a_2 < -3$. От (1) следва, че редицата е растяща и по индукция от (1), (2) и (3) следва, че
редицата $\{a_n\}_{n = 1}^\infty$ e монотоно растяща и ограничена отгоре от $-3$, тоест $ a_k \underset{k \to \infty}{\to} -3$. \\\\

Нека $\lambda = -6$. Тогава от (2) следва, че $a_2 = -6$ и отново от (2) следва, че $a_3 = -6$. Тогава по индукция, следва че $\forall n \in \N \; a_n = -6$. От където следва, че $a_k \underset{k \to \infty}{\to} -6$. \\\\

Нека $\lambda \in (-\infty, \; -6)$. Тогава от (1) следва, че редицата $\{a_n\}_{n = 1}^\infty$ e монотоно намаляваща, защото всеки член е строго по-малък от предишния.
От (2) и (3) следва, че $a_2 < -6 < -3$. Тогава $\forall n \in \N \; a_n < -6 < -3$, тоест редицата е неограничена откъдето следва, че $a_k \underset{k \to \infty}{\to} -\infty$. \\\\

Отговор: $\lim_{n \to \infty} a_n = \begin{cases}
    -\infty &, \; \lambda \in (-\infty, \; -6) \cup (3, \; 9) \\
    -6 &, \; \lambda \in \{-6, \; 3\} \\
    -3 &, \; \lambda \in (-6, \; 3) \\
    \infty &, \; \lambda \in (9, \; \infty)
\end{cases}$
\end{document}
\documentclass[a4paper,14pt]{extarticle}
\usepackage[left=3cm,right=3cm,top=1cm,bottom=2cm]{geometry}
\usepackage{amsmath,amsthm}
\usepackage{lipsum}
\usepackage{amssymb}
\usepackage{stmaryrd}
\usepackage[T1,T2A]{fontenc}
\usepackage[utf8]{inputenc}
\usepackage[bulgarian]{babel}
\usepackage[normalem]{ulem}
        
\newcommand{\R}{\mathbb{R}}
\newcommand{\N}{\mathbb{N}}
    
\setlength{\parindent}{0mm}
            
\title{Курсова задача №4}
\author{Иво Стратев, Информатика, 2-ри курс, група 3, фак. № 45342}
            
\begin{document}
\maketitle
    
Намерете във вид на степенен ред около т. $0$ решението на задача на Коши за даденото линейно
ДУ от 2-ри ред и определете радиуса на сходимост на реда:
\begin{align*}
    \begin{cases}
        y'' + 5xy' + 3 = 0 \\
        y(0) = -5 \\
        y'(0) = -3
    \end{cases}
\end{align*}

\section*{Решение:}
Решението на дадената задача на Коши търсим във вид на степенен ред около т. $0$. Тогава нека
\begin{align*}
    \forall x \in \R \; y(x) = \displaystyle\sum_{n = 0}^\infty a_n x^n
\end{align*}
За да намерим развитието на първата производна на $y$ диференцираме от двете страни и получаваме
\begin{align*}
    \forall x \in \R \; y'(x) = \left(\displaystyle\sum_{n = 0}^\infty a_n x^n\right)' = \displaystyle\sum_{n = 0}^\infty a_n (x^n)' = \displaystyle\sum_{n = 0}^\infty n a_n x^{n - 1}
\end{align*}
По аналогичен начин намираме развитието на втората производна на $y$
\begin{align*}
    \forall x \in \R \; y''(x) = \left(\displaystyle\sum_{n = 0}^\infty n a_n x^{n - 1}\right)' = \displaystyle\sum_{n = 1}^\infty n a_n (x^{n - 1})' = \displaystyle\sum_{n = 2}^\infty n (n - 1) a_n x^{n - 2}
\end{align*}
Сменяме индексите
\begin{align*}
    \forall x \in \R \; y''(x) = \displaystyle\sum_{n = 2}^\infty n (n - 1) a_{n + 2} x^{n - 2} = \displaystyle\sum_{n = 0}^\infty (n + 2) (n + 1) a_n x^n
\end{align*}
Заместваме в даденото линейно ДУ
\begin{align*}
    \forall x \in \R \;  0 = \displaystyle\sum_{n = 0}^\infty (n + 2) (n + 1) a_{n + 2} x^n + 5x\displaystyle\sum_{n = 0}^\infty n a_n x^{n - 1} + 3\displaystyle\sum_{n = 0}^\infty a_n x^n = \\\\
    = \displaystyle\sum_{n = 0}^\infty (n + 2) (n + 1) a_{n + 2} x^n + \displaystyle\sum_{n = 0}^\infty 5 n a_n x^n + \displaystyle\sum_{n = 0}^\infty 3 a_n x^n = \\\\
    = \displaystyle\sum_{n = 0}^\infty [(n + 2) (n + 1) a_{n + 2} + (5n + 3)a_n] x^n \implies \\\\
    \forall n \in \N \; (n + 2) (n + 1) a_{n + 2} + (5n + 3)a_n = 0 \implies \\\\
    \forall n \in \N \; a_{n + 2} = -\frac{5n + 3}{(n + 2) (n + 1)}a_n.
\end{align*}
Получихме, че за да имаме решение на даденото линейно ДУ във вид на степенен ред около т. $0$,
то членовете на този ред трябва да изпълняват рекуретното отношение
\begin{align*}
    \forall n \in \N \; a_{n + 2} = -\frac{5n + 3}{(n + 2) (n + 1)}a_n
\end{align*}
От началното условие $y(0) = -5$ получаваме
\begin{align*}
    -5 = y(0) = \displaystyle\sum_{n = 0}^\infty a_n 0^n = a_0.
\end{align*}
Следователно намерихме, че $a_0 = -5$. По аналогичен начин използвайки условието $y'(0) = -3$ получаваме
\begin{align*}
    -3 = y'(0) = \displaystyle\sum_{n = 1}^\infty n a_n 0^{n - 1} = a_1.
\end{align*}
Следователно $a_1 = -3$. Тогава членовете на решението на поставената задача на Коши във вид на ред около т. $0$
трябва да изпълняват следните условия
\begin{align*}
    a_0 = -5 \\
    a_1 = -3 \\
    \forall n \in \N \quad a_{n + 2} = -\frac{5n + 3}{(n + 2) (n + 1)}a_n.
\end{align*}
Пресмятеме първите няколко члена
\begin{align*}
    a_2 = -\frac{5.0 + 3}{(0 + 2)(0 + 1)}a_0 = -\frac{3}{2!}(-5) = \frac{15}{2!} \\\\
    a_3 = -\frac{5.1 + 3}{(1 + 2)(1 + 1)}a_1 = -\frac{8}{3.2}(-3) = \frac{8.3}{3!} \\\\
    a_4 = -\frac{5.2 + 3}{(2 + 2)(2 + 1)}a_2 = -\frac{13}{4.3}\frac{15}{2!} = -\frac{13.3.5}{4!} \\\\
    a_5 = -\frac{5.3 + 3}{(3 + 2)(3 + 1)}a_3 = -\frac{18}{5.4}\frac{8.3}{3!} = -\frac{18.8.3}{5!} \\\\
    a_6 = -\frac{5.4 + 3}{(4 + 2)(4 + 1)}a_4 = -\frac{23}{6.5}-\frac{13.3.5}{4!} = \frac{23.13.3.5}{6!} \\\\
    a_7 = -\frac{5.5 + 3}{(5 + 2)(5 + 1)}a_5 = -\frac{28}{7.6}-\frac{18.8.3}{5!} = \frac{28.18.8.3}{7!}
\end{align*}
Лесно се съобразява, че
\begin{align*}
    \forall n \in \N \; a_{2n} = 5\frac{(-1)^{n + 1}}{(2n)!}\displaystyle\prod_{k = 0}^{n - 1}(5(2k) + 3) = 5\frac{(-1)^{n + 1}}{(2n)!}\displaystyle\prod_{k = 0}^{n - 1}(10k + 3) \\\\
    \forall n \in \N \; a_{2n + 1} = 3\frac{(-1)^{n + 1}}{(2n + 1)!}\displaystyle\prod_{k = 0}^{n - 1}(5(2k + 1) + 3) = 3\frac{(-1)^{n + 1}}{(2n + 1)!}\displaystyle\prod_{k = 0}^{n - 1}(10k + 8)
\end{align*}
Тогава решението на дадената задача на Коши във вид на степенен ред около точката $0$ е
\begin{align*}
    \forall x \in \R \; y(x) = \displaystyle\sum_{n = 0}^\infty a_n x^n = \displaystyle\sum_{n = 0}^\infty a_{2n} x^{2n} + \displaystyle\sum_{n = 0}^\infty a_{2n + 1} x^{2n + 1} \\\\
    \forall n \in \N \quad a_n = \begin{cases}
        5\displaystyle\frac{(-1)^{\frac{n}{2} + 1}}{n!}\displaystyle\prod_{k = 0}^{\frac{n}{2} - 1}(10k + 3) &, \; n \equiv 0 \; (mod \; 2) \\\\
        3\displaystyle\frac{(-1)^{\frac{n - 1}{2} + 1}}{n!}\displaystyle\prod_{k = 0}^{\frac{n - 1}{2} - 1}(10k + 8) &, \; n \equiv 1 \; (mod \; 2)
    \end{cases}
\end{align*}
За да намерим радиуса на сходимост на реда, решение на задачата ще разгледаме по отделно реда на коефициентите с четен индекс и реда на коефициентите с нечетен.
Разглеждаме степенния ред само на членовете с четен индекс
\begin{align*}
    \displaystyle\sum_{n = 0}^\infty a_{2n} x^{2n} = \displaystyle\sum_{n = 0}^\infty 5\frac{(-1)^{n + 1}}{(2n)!}\displaystyle\prod_{k = 0}^{n - 1}(10k + 3) x^{2n}
\end{align*}
За да определим радиуса на сходимост на този ред пресмятаме границата
\begin{align*}
    \displaystyle\lim_{n \to \infty} \left|\displaystyle\frac{5(-1)^{n + 1}\displaystyle\prod_{k = 0}^{n - 1}(10k + 3)(2n + 2)!}{(2n)!5(-1)^{n + 2}\displaystyle\prod_{k = 0}^{n}(10k + 3)}\right| = \\\\
    = \displaystyle\lim_{n \to \infty} \displaystyle\frac{(2n + 2)(2n + 1)}{10n + 3} = \displaystyle\lim_{n \to \infty} \displaystyle\frac{4n^2 + 6n + 2}{10n + 3} = \\\\
    = \displaystyle\lim_{n \to \infty} \displaystyle\frac{4n^2\left(1 + \frac{3}{2n} + \frac{1}{2n^2}\right)}{10n\left(1 + \frac{3}{10n}\right)} = \infty
\end{align*}
Следователно
\begin{align*}
    \forall x \in \R \; \exists \displaystyle\lim_{n \to \infty}\displaystyle\sum_{n = 0}^\infty a_{2n} x^{2n}.
\end{align*}
Разглеждаме степенния ред само на членовете с нечетен индекс
\begin{align*}
    \displaystyle\sum_{n = 0}^\infty a_{2n + 1} x^{2n + 1} = \displaystyle\sum_{n = 0}^\infty 3\frac{(-1)^{n + 1}}{(2n + 1)!}\displaystyle\prod_{k = 0}^{n - 1}(10k + 8) x^{2n + 1}
\end{align*}
За да определим радиуса на сходимост на този ред пресмятаме границата
\begin{align*}
    \displaystyle\lim_{n \to \infty} \left|\displaystyle\frac{3(-1)^{n + 1}\displaystyle\prod_{k = 0}^{n - 1}(10k + 8)(2n + 3)!}{(2n + 1)!3(-1)^{n + 3}\displaystyle\prod_{k = 0}^{n}(10k + 8)}\right| = \\\\
    = \displaystyle\lim_{n \to \infty} \displaystyle\frac{(2n + 3)(2n + 2)}{10n + 8} = \displaystyle\lim_{n \to \infty} \displaystyle\frac{4n^2 + 8n + 6}{10n + 8} = \\\\
    = \displaystyle\lim_{n \to \infty} \displaystyle\frac{4n^2\left(1 + \frac{2}{n} + \frac{3}{2n^2}\right)}{10n\left(1 + \frac{4}{5n}\right)} = \infty
\end{align*}
Следователно
\begin{align*}
    \forall x \in \R \; \exists \displaystyle\lim_{n \to \infty}\displaystyle\sum_{n = 0}^\infty a_{2n + 1} x^{2n + 1}.
\end{align*}
Получихме, че за всяко $x \in \R$ двата реда за сходящи. Така получаваме че
\begin{align*}
    \forall x \in \R \; \displaystyle\lim_{n \to \infty}\displaystyle\sum_{n = 0}^\infty a_n x^n =
    \displaystyle\lim_{n \to \infty}\displaystyle\sum_{n = 0}^\infty a_{2n} x^{2n}
    + \displaystyle\lim_{n \to \infty}\displaystyle\sum_{n = 0}^\infty a_{2n + 1} x^{2n + 1} \\\\
    \implies \forall x \in \R \; \exists \displaystyle\lim_{n \to \infty}\displaystyle\sum_{n = 0}^\infty a_n x^n.
\end{align*}
Тоест получихме, че радиуса на сходимост реда около точката $0$ на даденото линейно ДУ е безкрайност.
\section*{Отговори:}
Решението във вид на степенен ред около т. $0$ на дадената задачата на Коши за линейно ДУ от 2-ри ред
\begin{align*}
    \begin{cases}
        y'' + 5xy' + 3 = 0 \\
        y(0) = -5 \\
        y'(0) = -3
    \end{cases}
\end{align*} е реда
\begin{align*}
    \forall x \in \R \; y(x) = \displaystyle\sum_{n = 0}^\infty a_n x^n = \displaystyle\sum_{n = 0}^\infty a_{2n} x^{2n} + \displaystyle\sum_{n = 0}^\infty a_{2n + 1} x^{2n + 1} = \\\\
    = \displaystyle\sum_{n = 0}^\infty 5\frac{(-1)^{n + 1}}{(2n)!}\displaystyle\prod_{k = 0}^{n - 1}(10k + 3) x^{2n}
    + \displaystyle\sum_{n = 0}^\infty 3\frac{(-1)^{n + 1}}{(2n + 1)!}\displaystyle\prod_{k = 0}^{n - 1}(10k + 8) x^{2n + 1}.
\end{align*}
Радиуса на сходимост на този ред е безкрайност. Тоест реда е сходящ за всяка стойност на аргумента.
\end{document}